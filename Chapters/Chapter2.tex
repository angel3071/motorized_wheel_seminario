% Chapter 2

\chapter{Planeaci\'on del producto} % Main chapter title

\label{Chapter2} % For referencing the chapter elsewhere, use \ref{Chapter1} 

%----------------------------------------------------------------------------------------

% Define some commands to keep the formatting separated from the content 

%----------------------------------------------------------------------------------------

\section{Mercado}
El proceso de planeaci\'on empieza con una identificaci\'on de oportunidades para el desarrollo de un producto. Estas oportunidades pueden abarcar cualquiera de los cuatro tipos de proyectos que se describen a continuaci\'on.\\

Los proyectos de desarrollo de productos se pueden clasificar en:

\begin{itemize}
	\item Nuevas plataformas de productos: Este tipo de proyecto comprende un gran esfuerzo de desarrollo para crear una nueva familia de productos basados en una plataforma com\'un. La familia del nuevo producto abordar\'ia mercados y categor\'ias de productos ya conocidos.
	\item Derivados de productos existente: Estos proyectos ampl\'ian una plataforma de productos ya existentes para satisfacer mejor los mercados conocidos con uno o m\'as productos nuevos. 
	\item Mejoras incrementales a productos existentes: En estos proyectos solo se agregan o modifican algunas funciones a productos existentes para mantener actualizada y competitiva la l\'inea del producto.
	\item Productos fundamentalmente nuevos: Estos proyectos abarcan tecnolog\'ias radicalmente nuevas de producci\'on o de producto y pueden ayudar a entrar en mercados nuevos y desconocidos. Estos proyectos involucran en forma inherente m\'as riesgo; no obstante, el \'exito a largo plazo de la empresa puede depender de lo que se aprende en estos importantes proyectos.
\end{itemize}

El tipo de proyecto que comprende este documento ser\'a el de mejoras incrementales a productos existentes, una vez identificado este segmento se realiz\'o la revisi\'on a detalle sobre el primer paso del proceso de planeación del producto.

\begin{figure}[th]
	\centering
	\includegraphics[width=.8\textwidth]{Figures/elproceso.png}
	\decoRule
	\caption{Proceso de planeaci\'on de un producto \parencite{desarrollo}}
	\label{fig:elproceso}
\end{figure}

\subsection{Discapacidad en M\'exico}

De acuerdo con la Clasificaci\'on Internacional del Funcionamiento, de la Discapacidad y de la Salud, presentada en 2001, las personas con discapacidad “son aquellas que tienen una o más deficiencias f\'isicas, mentales, intelectuales o sensoriales y que al interactuar con distintos ambientes del entorno social pueden impedir su participaci\'on plena y efectiva en igualdad de condiciones a las dem\'as”.

Al año 2010, las personas en M\'exico que tienen alg\'un tipo de discapacidad son 5 millones 739 mil 270, lo que representa 5.1\% de la población total.\\
Una de las dificultades m\'as importantes a nivel nacional es la de caminar o moverse. Que referencia a la dificultad de una persona para moverse, caminar, desplazarse o subir escaleras debido a la falta de toda o una parte de sus piernas; incluye tambi\'en a quienes teniendo sus piernas no tienen movimiento o presentan restricciones para moverse, de tal forma que necesitan ayuda de otras persona, silla de ruedas u otro aparato, como andadera o pierna artificial.

\begin{figure}[th]
	\centering
	\includegraphics[width=.8\textwidth]{Figures/dificultad.png}
	\decoRule
	\caption{Porcentaje de la poblaci\'on con discapacidad seg\'un dificultad en la actividad (Año 2010) \parencite{discapacidad}}
    \label{fig:dificultad}
\end{figure}

Algunos otros datos estad\'isticos sobre discapacidad en M\'exico que nos permitir\'an segmentar mejor el mercado para poder ofrecer un producto de valor para la poblaci\'on afectada por esta dificultad son:

\begin{itemize}
\item Las dificultades para caminar y para ver son las m\'as reportadas entre las personas con discapacidad.
\item Los principales detonantes de discapacidad en el pa\'is son las enfermedades (41.3\%) y la edad avanzada (33.1\%).
\item 23.1\% de la poblaci\'on con discapacidad de 15 años y m\'as no cuentan con alg\'un nivel de escolaridad.
\item De la poblaci\'on con discapacidad, 83.3\% es derechohabientes o est\'a afiliada a servicios de salud.
\item Participa en actividades económica 39.1\% de la poblaci\'on con discapacidad de 15 años y más, frente a 64.7\% de su contraparte sin discapacidad
\end{itemize}


\section{Modelo de negocio}

Para poder desarrollar el modelo de negocio para este producto nos basamos en el modelo de negocio Canvas desarrollado por Alexander Osterwalder y Yves Pigneur. El cual constituye una herramienta esencial para crear modelos de negocios, describiendo diferentes aspectos de la idea de negocio necesarios para el correcto funcionamiento de un proyecto.\\

La herramienta toma en cuenta distintos apartados interrelacionados que responden a cuestiones comunes como qu\'e es lo que hacemos, c\'omo lo hacemos y a qui\'en nos dirigimos, cubriendo as\'i todos los aspectos b\'asicos de un negocio: segmentos de clientes, propuesta de valor, canales, relaci\'on con el clientes, fuentes de ingresos, recursos clave, actividades clave, socios clave y estructura de costes.\\

A continuaci\'on se presenta el trabajo desarrollado para lograr la definici\'on del modelo de negocio para la silla de ruedas motorizada.

\begin{enumerate}
    \item \textbf{Segmento de clientes}\\
Personas con dificultad para caminar, con falta de toda o una parte de sus piernas \'o  adultos mayores con dificultades para caminar. Que se encuentren en un rango de edad de 5 a 80 años. Con ingresos econ\'omicos limitados.

    \item \textbf{Propuesta de valor}\\
Es un producto que se puede instalar en la mayor\'ia de las sillas convencionales, es de costo accesible, proporciona una experiencia m\'as satisfactoria e incrementa los niveles de seguridad para el usuario.

    \item \textbf{Canales de distribuci\'on}\\
La compra se puede realizar acudiendo directamente a los centros de venta, a trav\'es de los distribuidores o por internet, sin embargo para compras por internet deber\'a acudir al centro de venta para que se realice el montaje del producto.

    \item \textbf{Relaci\'on con el cliente}\\
El medio masivo para promocionar el producto ser\'a a trav\'es de redes sociales y la p\'agina web, adicional se buscar\'an alianzas con empresas que venden equipo medico para que ofrezcan tambi\'en nuestro producto.

    \item \textbf{Fuentes de ingreso}\\
Venta del m\'odulo de asistencia motriz y servicios de instalaci\'on

    \item \textbf{Recursos clave}
        \begin{itemize}
            \item Hardware: motor, bater\'ia, electr\'onica, accesorios de montaje.
            \item Centro de venta e instalaci\'on
            \item T\'ecnicos de instalaci\'on del m\'odulo
        \end{itemize}

    \item \textbf{Actividades clave}\\
Producci\'on: Se comprometieron entregas peri\'odicas de los insumos acorde a la demanda del mes siguiente y adicional se tiene un stock capaz de soportar el desabasto de componentes primarios del m\'odulo de asistencia motriz por lo que se tendr\'ia hasta dos meses de producción asegurada.

    \item \textbf{Tus socios clave}\\
Proveedores de batería, motor, sensores y electron\'ica.

\end{enumerate}
