% Chapter 2

\chapter{Conclusiones} % Main chapter title

\label{Chapter5} % For referencing the chapter elsewhere, use \ref{Chapter1} 

%----------------------------------------------------------------------------------------

% Define some commands to keep the formatting separated from the content 

%----------------------------------------------------------------------------------------

\section{Resultados obtenidos}
La investigaci\'on realizada para poder sentar las bases para el desarrollo de un producto como lo es la silla de ruedas motorizada nos deja reflexiones muy significativas. Por una parte el reconocimiento de la poblaci\'on discapacitada del pa\'is que hoy se encuentra con muchas dificultades para poder desenvolverse en la sociedad y que dadas las condiciones socioecon\'omicas, de desarrollo y de salud del pa\'is se ven potencializados los retos a los cuales esta poblaci\'on se tiene que enfrentar a fin de lograr calidad de vida y precisamente a este punto la relevancia de nuestra profesi\'on como desarrolladores de tecnolog\'ia al servicio de la patria.//
La segunda reflexi\'on es en el aspecto de nuestra formaci\'on, de la cual estamos completamente orgullosos y satisfechos pues como futuros egresados de la Facultad de Ingenier\'ia y de la carrera de Ingenier\'ia Mecatr\'onica hoy tenemos la capacidad de planificar, diseñar y desarrollar un producto integral que genere valor para la sociedad.//
Los resultados obtenidos en este trabajo nos motivan a creer que tenemos la capacidad para desarrollar tecnolog\'ia y soluciones ingenieriles cien por ciento mexicanas.
